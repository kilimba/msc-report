\cleardoublepage
%\begin{vcentrepage}
%\noindent\rule[2pt]{\textwidth}{0.8pt}\\
\begin{center}
{\Large\textbf{ABSTRACT}}
\end{center}
\renewcommand{\baselinestretch}{1.50}\normalsize
A lot of work has been done on spatial visualizations for public health.  The feasibility of creating poverty maps for Indonesia at various administrative levels to help with the implementation of programs which target the poor was investigated in [6]. Their focus however was on a new methodology for imputing per capita consumption for each household in the population based on data collected from household surveys and data collected from population censuses. Their report however did not focus much on how the visualization platform was to be built. The final product though, after all the computations is the visualization, the poverty map of the country. 



%\noindent\rule[2pt]{\textwidth}{0.8pt}
%\end{vcentrepage}
