\chapter{Background and Related Works}\label{chap2}
\textit{This chapter gives the background and related works that have been done in the field of public health data visualization}
%\vspace{2ex}\vfill
%\minitoc
%\newpage
%\renewcommand{\baselinestretch}{1.80}\normalsize

\vfill
\minitoc
\newpage
\renewcommand{\baselinestretch}{1.80}\normalsize
The topic of visualization of public health data was identified in 2009 by the CDC as one of six major concerns which must be addressed by the public health community in order to advance public health surveillance in the 21st century [cite 9]. However, very little has been researched in terms of standardization of the workflow and linking technologies for heterogeneous data sources needed specifically for visualizations in public health science [cite 10].

Of crucial importance when dealing with large datasets is the need for the users of the data, be they scientists or any other stakeholders to be able to discover the relations among and between the results of data analyses and queries [cite 10]. However, due to bottlenecks resulting from resource cost and lack of required skills, data visualization becomes an end product of scientific analysis rather than an exploration tool which facilitates scientists to generate better hypotheses in the continually more data-intensive scientific process [cite 10]. The use of such visualizations are usually utilised in business analytics, open government data systems, and media infographics but have generally not been used in public health. However, the capabilities currently being seen by web-based tools and technologies may be the breakthrough in resolving the scientific visualization bottleneck.

A lot of work has been done on spatial visualizations for public health [cite 4,6,7].  The feasibility of creating poverty maps for Indonesia at various administrative levels to help with the implementation of programs which target the poor was investigated in [cite 6]. Their focus however was on a new methodology for imputing per capita consumption for each household in the population based on data collected from household surveys and data collected from population censuses. Additionally, their report did not focus much on how the visualization platform was to be built. The final product though, after all the computations is the visualization, the poverty map of the country.  

Along a similar spatial theme, [cite 7] looked at how the agents of parasitic diseases are spatially distributed using map visualizations. The tools of interest used in their research were the two closely linked Google products, Google Earth and Google Maps. Though they provide a little more implementation details, the mix of tools used are not all open-source.

Other research on implementing a data visualization platform for community health assessment (CHA) used open source technologies but was limited to a desktop application and could not be accessed online [cite 11]. Web based tools have been seen as the preferred platform of choice for public health researchers as they permit distributed access, reduced software implementation costs and wider exposure of public health information for public dissemination [cite 4,12,13]. The report also gives scant details of the actual open source tools used and how they were put together in a way which would allow recreation of the steps. 

The use of multi-panel graphs to illustrate trends and anomalies which would otherwise be obscured by traditional epidemiological visualization techniques such as pyramids and time-series plots was explored in [cite 8]. Under future developments in their report however, they acknowledge that two other features if incorporated to these graphs would enhance their impact, namely the dynamic display of data and interactivity.

Existing tools offer a range of features and functions to allow for exploration, analysis and visualization of public health users data, but the tools are often for siloed applications incapable of reciprocal operation with other, related information systems. They are isolated to the jurisdictions and organisations which developed them limiting their widespread adoption by other agencies or organisations[cite 13]. Interoperability of the visualization tools has been identified in a systematic literature review of infectious disease visualization tools as a prominent theme, due to challenges associated with increasingly collaborative and interdisciplinary nature of disease surveillance, control and prevention [cite 13]. 

The CDC’s inclusion of VA as one of its six areas of focused research in public health lends credence to the importance of this research project. Furthermore there is little that has been researched in terms of standardisation of the workflow, tools and linking technologies for visualizations in the public health domain, as well as few existing web based tools for health related data visualization [cite 4,10].  This research will add to the body of literature broadly in the subject of tools and technologies for data visualization in general, and specifically for implementing a data visualization platform which is generalizable for all INDEPTH HDSS sites with the aim of allowing it to become a central piece of the scientific process. It will also augment on work already done [cite 8] by incorporating dynamic display of data and interactivity into multi-panel graphs for epidemiological research [cite 8].
