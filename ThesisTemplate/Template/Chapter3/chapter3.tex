\chapter{Research Aims and Objectives}\label{chap3}
\textit{In this chapter we will elaborate on the problem we are addressing. We shall also discuss how this problem is relevant to Health and Demographic Surveillance Sites (HDSS's) and our motivation to address it}
\vspace{2ex}\vfill
\minitoc
\newpage
 
\section{Problem Statement}\label{prob}
One critical requirement for successful public health surveillance is the ability to analyse and present data so that it is understandable to a range of public health stakeholders. In public health, VA can be viewed as the bridge between the availability of surveillance data in database architectures and useful information derived from this available data [cite 9].

INDEPTH Health and Demographic Surveillance Sites (HDSS's) deal with complex longitudinal data and, as a result, knowledge transfer to stakeholders is challenging. Better visualisation of this data is therefore required in order for potential scientific users to maximise exploratory data analysis and hypothesis generation. It would also aid decision-makers and the society at large to visualise this information in terms understandable by them. Such a visualization tool will also improve field work research activities by providing summary data of operational progress, e.g. fieldwork data collection progress, data entry progress, data archiving progress or other parameters such as data quality. This will serve to improve operational decision making and data quality. However, datasets at HDSS sites are normally under-visualised. These HDSS sites currently have no generalizable framework for implementing a data visualization platform to be used at these sites.

\section{Motivation}\label{motivation}

The current under-visualization of HDSS datasets shows little promise of improvement in a harmonised way (across multiple sites) unless specific research efforts are directed towards finding a generalizable solution for delivering interactive visualizations, supporting exploratory analyses and real-time displays of operational progress. Furthermore, there is a paucity of research specifically on the technologies and tools which can be used to create such a data visualization [cite 10].

\section{Overall Aim}\label{aim}

The overall aim of the project was to increase the utilization of data in INDEPTH sites through interactive data visualization. This was aimed at improving hypotheses generation at these research sites as well as increase operational awareness.

\section{Specific Objectives}\label{objectives}
The specific objectives of this research were:

\begin{enumerate}
 \item To design a data visualization platform for the Africa Centre for Health and Population Studies (ACHPS).
 \item To build a data visualization platform for ACHPS in order to increase data utilization and hypotheses generation at the site
 \item To create a developer manual for data visualization so that the process for building the platform can be reproduced.
 \item iv. To evaluate the usability of the developed platform for easy integration into the operational research cycle of the site
\end{enumerate}